\documentclass[12pt]{article}
 
\usepackage[margin=1in]{geometry}
\usepackage{amsmath,amsthm,amssymb}
\usepackage{mathtools}
\DeclarePairedDelimiter{\ceil}{\lceil}{\rceil}
%\usepackage{mathptmx}
\usepackage{accents}
\usepackage{comment}
\usepackage{graphicx}
\usepackage{IEEEtrantools}
 \usepackage{float}
 
\newcommand{\N}{\mathbb{N}}
\newcommand{\Z}{\mathbb{Z}}
\newcommand{\R}{\mathbb{R}}
\newcommand{\Q}{\mathbb{Q}}
\newcommand*\conj[1]{\bar{#1}}
\newcommand*\mean[1]{\bar{#1}}
\newcommand\widebar[1]{\mathop{\overline{#1}}}


\newcommand{\cc}{{\mathbb C}}
\newcommand{\rr}{{\mathbb R}}
\newcommand{\qq}{{\mathbb Q}}
\newcommand{\nn}{\mathbb N}
\newcommand{\zz}{\mathbb Z}
\newcommand{\aaa}{{\mathcal A}}
\newcommand{\bbb}{{\mathcal B}}
\newcommand{\rrr}{{\mathcal R}}
\newcommand{\fff}{{\mathcal F}}
\newcommand{\ppp}{{\mathcal P}}
\newcommand{\eps}{\varepsilon}
\newcommand{\vv}{{\mathbf v}}
\newcommand{\ww}{{\mathbf w}}
\newcommand{\xx}{{\mathbf x}}
\newcommand{\ds}{\displaystyle}
\newcommand{\Om}{\Omega}
\newcommand{\dd}{\mathop{}\,\mathrm{d}}
\newcommand{\ud}{\, \mathrm{d}}
\newcommand{\seq}[1]{\left\{#1\right\}_{n=1}^\infty}
\newcommand{\isp}[1]{\quad\text{#1}\quad}

\DeclareMathOperator{\imag}{Im}
\DeclareMathOperator{\re}{Re}
\DeclareMathOperator{\diam}{diam}
\DeclareMathOperator{\Tr}{Tr}

\def\upint{\mathchoice%
    {\mkern13mu\overline{\vphantom{\intop}\mkern7mu}\mkern-20mu}%
    {\mkern7mu\overline{\vphantom{\intop}\mkern7mu}\mkern-14mu}%
    {\mkern7mu\overline{\vphantom{\intop}\mkern7mu}\mkern-14mu}%
    {\mkern7mu\overline{\vphantom{\intop}\mkern7mu}\mkern-14mu}%
  \int}
\def\lowint{\mkern3mu\underline{\vphantom{\intop}\mkern7mu}\mkern-10mu\int}




\newenvironment{theorem}[2][Theorem]{\begin{trivlist}
\item[\hskip \labelsep {\bfseries #1}\hskip \labelsep {\bfseries #2.}]}{\end{trivlist}}
\newenvironment{lemma}[2][Lemma]{\begin{trivlist}
\item[\hskip \labelsep {\bfseries #1}\hskip \labelsep {\bfseries #2.}]}{\end{trivlist}}
\newenvironment{exercise}[2][Exercise]{\begin{trivlist}
\item[\hskip \labelsep {\bfseries #1}\hskip \labelsep {\bfseries #2.}]}{\end{trivlist}}
\newenvironment{problem}[2][Problem]{\begin{trivlist}
\item[\hskip \labelsep {\bfseries #1}\hskip \labelsep {\bfseries #2.}]}{\end{trivlist}}
\newenvironment{question}[2][Question]{\begin{trivlist}
\item[\hskip \labelsep {\bfseries #1}\hskip \labelsep {\bfseries #2.}]}{\end{trivlist}}
\newenvironment{corollary}[2][Corollary]{\begin{trivlist}
\item[\hskip \labelsep {\bfseries #1}\hskip \labelsep {\bfseries #2.}]}{\end{trivlist}}

\newenvironment{solution}{\begin{proof}[Solution]}{\end{proof}}
 
\begin{document}
 
% --------------------------------------------------------------
%                         Start here
% --------------------------------------------------------------
\title{Math 115A Homework 3}
\author{Ethan Martirosyan}
\date{\today}
\maketitle
\hbadness=99999
\hfuzz=50pt
\section*{Problem 1}
We claim that the function
\[
\frac{\mu(d)}{d}
\] is multiplicative. Let us suppose that $m$ and $n$ are positive integers such that $(m,n) = 1$. Notice that
\[
 \frac{\mu(mn)}{mn} = \frac{\mu(m)\mu(n)}{mn} = \frac{\mu(m)}{m} \cdot \frac{\mu(n)}{n}
\] since $\mu$ is multiplicative. In class, we proved that if a function is multiplicative, so is its mobius transform. Thus, the function 
\[
g(n) = \sum_{d \mid n} \frac{\mu(d)}{d}
\] is multiplicative. Since $n = p_1^{e_1} \cdots p_k^{e_k}$, we have
\[
g(n) = g(p_1^{e_1} \cdots p_k^{e_k}) = g(p_1^{e_1}) \cdots g(p_k^{e_k})
\] For any $i$, we have
\[
g(p_i^{e_i}) = \sum_{d \mid p_i^{e_i}} \frac{\mu(d)}{d} = \sum_{j = 0}^{e_i} \frac{\mu(p_i^j)}{p_i^j} 
\] If $j > 1$, then
\[
\frac{\mu(p_i^j)}{p_i^j} = 0
\] by the definition of $\mu$. If $j = 1$, then
\[
\frac{\mu(p_i^j)}{p_i^j} = \frac{\mu(p_i)}{p_i} = -\frac{1}{p_i}
\] If $j = 0$, then
\[
\frac{\mu(p_i^j)}{p_i^j} = \frac{\mu(p_i^0)}{p_i^0} = \frac{\mu(1)}{1} = 1
\] Thus, we find that
\[
g(p_i^{e_i}) = \sum_{j = 0}^{e_i} \frac{\mu(p_i^j)}{p_i^j}  = \bigg(1 - \frac{1}{p_i}\bigg)
\] Therefore, we may deduce that
\[
\sum_{d \mid n} \frac{\mu(d)}{d} = g(n) = g(p_1^{e_1}) \cdots g(p_k^{e_k}) = \bigg(1 - \frac{1}{p_1}\bigg) \cdots \bigg(1 - \frac{1}{p_k}\bigg)
\]
\newpage
\section*{Problem 2}
To prove that the set
\[
T = \{am+b: \; m \in S\}
\] is a complete system of residues modulo $n$, we must show that $\vert T \vert = n$ and that $T$ is incongruent modulo $n$. To prove that $\vert T \vert = n$, we may define the function $f: S \rightarrow T$ as follows:
\[
f(m) = am+b
\] First, we claim that $f$ is injective. Suppose that $m_1, m_2 \in S$ and that
\[
f(m_1) = f(m_2)
\] or
\[
am_1 + b = am_2 + b
\] Since $a$ is nonzero, we find that
\[
m_1 = m_2
\] Thus $f$ is injective. Next, we claim that $f$ is surjective. Suppose that $am+b \in T$ where $m \in S$. Then, it is evident that
\[
f(m) = am + b
\] so that $f$ is surjective. We deduce that $f$ is bijective. Thus, we have $\vert T \vert = \vert S \vert = n$ (we know that $\vert S \vert = n$ since $S$ is a complete system of residues modulo $n$). Next, we claim that $T$ is incongruent modulo $n$. Suppose that
\[
am_1 + b \equiv am_2 + b \pmod{n}
\] Subtracting $b$ yields
\[
am_1 \equiv am_2 \pmod{n}
\] Since $(a,n) = 1$ by assumption, we may divide by $a$ to deduce that
\[
m_1 \equiv m_2 \pmod{n}
\] Since $m_1, m_2 \in S$ and $S$ is incongruent modulo $n$, we may deduce that $m_1 = m_2$ so that
\[
am_1 + b = am_2 + b
\] This proves that $T$ is incongruent modulo $n$, so we know that $T$ is a complete system of residues modulo $n$.
\newpage
\section*{Problem 3}
To prove that the set
\[
Y = \{am: \; m \in X\}
\] is a reduced system of residues modulo $n$, we must show that $\vert Y \vert = \varphi (n)$, that every element of $Y$ is coprime to $n$, and that $Y$ is incongruent modulo $n$. To show that $\vert Y \vert = \varphi(n)$, we may define the function $f: X \rightarrow Y$ as follows:
\[
f(m) = am
\] First, we claim that $f$ is injective. Let
\[
f(m_1) = f(m_2)
\] where $m_1, m_2 \in X$. Then we have
\[
am_1 = am_2
\] so that $m_1 = m_2$. Thus $f$ is injective. Next, we claim that $f$ is surjective. Let $am \in Y$, where $m \in X$. Then, we have
\[
f(m) = am
\] so that $f$ is surjective. Thus $f$ is bijective, and we know that $\vert Y \vert = \vert X \vert = \varphi(n)$ (we know that $\vert X \vert = \varphi(n)$ because $X$ is assumed to be a reduced system of residues modulo $n$). Next, we claim that every element of $Y$ is coprime to $n$. Let $am \in Y$, where $m\in X$. Notice that $(a,n) = 1$ by assumption and $(m,n) = 1$ since $X$ is a reduced system of residues modulo $n$. Thus, we find that $(am,n) = 1$. Finally, we claim that $Y$ is incongruent modulo $n$. Suppose that
\[
am_1 \equiv am_2 \pmod{n}
\] Since $(a,n) = 1$, we find that
\[
m_1 \equiv m_2 \pmod{n}
\] Because $X$ is incongruent modulo $n$, we know that $m_1 = m_2$ so that $am_1 = am_2$. This means that $Y$ is incongruent modulo $n$. Thus, we may deduce that $Y$ is a reduced system of residues modulo $n$.
\newpage
\section*{Problem 4}
First, we may let $n = 1$. Notice that $1^p - 1 = 0$ is a multiple of $p$, so we have
\[
1^p \equiv 1 \pmod{p}
\] Next, we may suppose that $n \geq 1$ and
\[
n^p \equiv n \pmod{p}
\] We claim that this congruence holds for $n+1$. Notice that
\[
(n+1)^p = \sum_{k=0}^p \binom{p}{k} n^k 1^{p-k} = \sum_{k=0}^p \binom{p}{k} n^k
\] Notice that
\[
\binom{p}{k} = \frac{p!}{k! (p-k)!}
\] We know that $p$ divides $p!$. For $1 \leq k \leq p - 1$, we claim that $p$ does not divide $k!(p-k)!$. If $p$ did divide $k!(p-k)!$, then $p$ must divide $k!$ or $(p-k)!$ because $p$ is prime. This means that $p$ must divide some number less than or equal to $k$ or some number less than or equal to $p-k$, which is not true. Thus, we find that
\[
p \mid \binom{p}{k}
\] for $1 \leq k \leq p - 1$. Finally, we obtain
\[
(n+1)^p = \sum_{k=0}^p \binom{p}{k} n^k \equiv 1 + n^p \equiv n + 1 \pmod{p}
\] since
\[
n^p \equiv n \pmod{p}
\] by our induction hypothesis.
\newpage
\section*{Problem 5}
First, we will compute $(256,337)$ as follows:
\begin{align*}
337 & = 256 \cdot 1 + 81 \\
256 & = 81 \cdot 3 + 13 \\
81 &= 13 \cdot 6 + 3\\
13 &= 3 \cdot 4 + 1 \\
3 & = 1 \cdot 3
\end{align*} Thus, we find that $(256,337) = 1$. That means that there is one solution to this congruence. To find it, we will first compute the inverse of $256$ modulo $337$ as follows:
\begin{align*}
& 1 = 13 - 3 \cdot 4 = 13 - (81 - 13 \cdot 6) \cdot 4 = 13 \cdot 25 + 81 \cdot -4 = (256 - 81 \cdot 3) \cdot 25 + 81 \cdot -4\\
& = 256 \cdot 25 + 81 \cdot -79 = 256 \cdot 25 + (337 - 256 \cdot 1) \cdot -79 = 256 \cdot 104 + 337 \cdot -79
\end{align*} Thus, we find that
\[
256 \cdot 104 \equiv 1 \pmod{337}
\] Multiplying both sides of the congruence 
\[
256x \equiv 179 \pmod{337}
\] by $104$ yields
\[
x \equiv 104 \cdot 179 \equiv 81 \pmod{337}
\]
\newpage
\section*{Problem 6}
First, we will compute $(1215,2755)$ as follows:
\begin{align*}
2755 &= 1215 \cdot 2 + 325\\
1215 &= 325 \cdot 3 + 240\\
325 &= 240 \cdot 1+ 85\\
240 &= 85 \cdot 2 + 70\\
85 &= 70 \cdot 1+ 15\\
70 &= 15 \cdot 4 + 10\\
15 &= 10 \cdot 1 + 5\\
10 &= 5 \cdot 2
\end{align*} Thus, we find that $(1215, 2755) = 5$. Since $5 \mid 560$, there are five solutions to this congruence. Now, we may divide the congruence
\[
1215 \equiv 560 \pmod{2755}
\] by $5$ to obtain
\[
243 x \equiv 112 \pmod{551}
\] To compute the inverse of $243$ modulo $551$, we must first perform the Euclidean Algorithm:
\begin{align*}
551 &= 243 \cdot 2 + 65 \\
243 & = 65 \cdot 3 + 48\\
65 & = 48 \cdot 1 + 17 \\
48 & = 17 \cdot 2+ 14 \\
17 & = 14 \cdot 1 + 3 \\
14 & = 3\cdot 4 + 2\\
3 & = 2 \cdot 1 + 1 \\
2 & = 2 \cdot 1
\end{align*} Then, we write
\begin{align*}
&1 = 3 - 2\cdot 1 = 3 - (14 - 3\cdot 4) \cdot 1 = 3 \cdot 5 + 14 \cdot - 1 = (17 - 14 \cdot 1) \cdot 5 + 14 \cdot -1 \\
& = 14 \cdot - 6 + 17 \cdot 5 = (48 - 17 \cdot 2) \cdot -6 + 17 \cdot 5 = 17 \cdot 17 + 48 \cdot -6\\
& = (65 - 48 \cdot 1) \cdot 17 + 48 \cdot -6 = 65 \cdot 17 + 48 \cdot -23 = 65\cdot 17 + (243 - 65 \cdot 3) \cdot -23 \\
& = 65 \cdot 86 + 243 \cdot -23 = (551 - 243 \cdot 2) \cdot 86 + 243 \cdot -23 = 551 \cdot 86 + 243 \cdot -195
\end{align*} Thus $-195$ is an inverse of $243$ modulo $551$. To obtain a positive inverse, we may add $551$ to $-195$ to obtain $356$. That is, we have
\[
356 \cdot 243 \equiv 1 \pmod{551}
\] We may multiply both sides of the congruence
\[
243x \equiv 112 \pmod{551}
\] by $356$ to obtain
\[
x \equiv 356 \cdot 112 \equiv 200 \pmod{551}
\] Thus, $200 \pmod{551}$ is one solution of the congruence $243 x \equiv 112 \pmod{551}$. It is also a solution of the congruence $1215 x \equiv 560 \pmod{2755}$. The four other solutions are obtained by adding multiples of $551$; they are
\[
200 + 551 \cdot 1,\, 200 + 551 \cdot 2,\, 200 + 551 \cdot 3,\, 200 + 551 \cdot 4 \pmod{2755}
\] Therefore, the five solutions of this congruence are
\[
200, \, 751,\, 1302,\, 1853,\, 2404 \pmod{2755}
\]
\end{document} 
\documentclass[12pt]{article}
 
\usepackage[margin=1in]{geometry}
\usepackage{amsmath,amsthm,amssymb}
\usepackage{mathtools}
\DeclarePairedDelimiter{\ceil}{\lceil}{\rceil}
%\usepackage{mathptmx}
\usepackage{accents}
\usepackage{comment}
\usepackage{graphicx}
\usepackage{IEEEtrantools}
 \usepackage{float}
 
\newcommand{\N}{\mathbb{N}}
\newcommand{\Z}{\mathbb{Z}}
\newcommand{\R}{\mathbb{R}}
\newcommand{\Q}{\mathbb{Q}}
\newcommand*\conj[1]{\bar{#1}}
\newcommand*\mean[1]{\bar{#1}}
\newcommand\widebar[1]{\mathop{\overline{#1}}}


\newcommand{\cc}{{\mathbb C}}
\newcommand{\rr}{{\mathbb R}}
\newcommand{\qq}{{\mathbb Q}}
\newcommand{\nn}{\mathbb N}
\newcommand{\zz}{\mathbb Z}
\newcommand{\aaa}{{\mathcal A}}
\newcommand{\bbb}{{\mathcal B}}
\newcommand{\rrr}{{\mathcal R}}
\newcommand{\fff}{{\mathcal F}}
\newcommand{\ppp}{{\mathcal P}}
\newcommand{\eps}{\varepsilon}
\newcommand{\vv}{{\mathbf v}}
\newcommand{\ww}{{\mathbf w}}
\newcommand{\xx}{{\mathbf x}}
\newcommand{\ds}{\displaystyle}
\newcommand{\Om}{\Omega}
\newcommand{\dd}{\mathop{}\,\mathrm{d}}
\newcommand{\ud}{\, \mathrm{d}}
\newcommand{\seq}[1]{\left\{#1\right\}_{n=1}^\infty}
\newcommand{\isp}[1]{\quad\text{#1}\quad}

\DeclareMathOperator{\imag}{Im}
\DeclareMathOperator{\re}{Re}
\DeclareMathOperator{\diam}{diam}
\DeclareMathOperator{\Tr}{Tr}

\def\upint{\mathchoice%
    {\mkern13mu\overline{\vphantom{\intop}\mkern7mu}\mkern-20mu}%
    {\mkern7mu\overline{\vphantom{\intop}\mkern7mu}\mkern-14mu}%
    {\mkern7mu\overline{\vphantom{\intop}\mkern7mu}\mkern-14mu}%
    {\mkern7mu\overline{\vphantom{\intop}\mkern7mu}\mkern-14mu}%
  \int}
\def\lowint{\mkern3mu\underline{\vphantom{\intop}\mkern7mu}\mkern-10mu\int}




\newenvironment{theorem}[2][Theorem]{\begin{trivlist}
\item[\hskip \labelsep {\bfseries #1}\hskip \labelsep {\bfseries #2.}]}{\end{trivlist}}
\newenvironment{lemma}[2][Lemma]{\begin{trivlist}
\item[\hskip \labelsep {\bfseries #1}\hskip \labelsep {\bfseries #2.}]}{\end{trivlist}}
\newenvironment{exercise}[2][Exercise]{\begin{trivlist}
\item[\hskip \labelsep {\bfseries #1}\hskip \labelsep {\bfseries #2.}]}{\end{trivlist}}
\newenvironment{problem}[2][Problem]{\begin{trivlist}
\item[\hskip \labelsep {\bfseries #1}\hskip \labelsep {\bfseries #2.}]}{\end{trivlist}}
\newenvironment{question}[2][Question]{\begin{trivlist}
\item[\hskip \labelsep {\bfseries #1}\hskip \labelsep {\bfseries #2.}]}{\end{trivlist}}
\newenvironment{corollary}[2][Corollary]{\begin{trivlist}
\item[\hskip \labelsep {\bfseries #1}\hskip \labelsep {\bfseries #2.}]}{\end{trivlist}}

\newenvironment{solution}{\begin{proof}[Solution]}{\end{proof}}
 
\begin{document}
 
% --------------------------------------------------------------
%                         Start here
% --------------------------------------------------------------
\title{Math 115A Homework 1}
\author{Ethan Martirosyan}
\date{\today}
\maketitle
\hbadness=99999
\hfuzz=50pt
\section*{Problem 1}
By the Fundamental Theorem of Arithmetic, we may write $c = p_1^{\beta_1}p_2^{\beta_2}\cdots p_k^{\beta_k}$ for some distinct primes $p_1,\ldots,p_k$ and $\beta_1,\ldots,\beta_k > 0$. Then $ab = p_1^{2\beta_1} \cdots p_k^{2\beta_k}$. Every prime $p_i$ must divide $a$ or $b$. To prove this, let us suppose that $p_i$ does not divide $a$. Then, we claim that $p_i$ must divide $b$. If $p_i$ does not divide $a$, then $(a,p_i) = 1$. Then, there must exist integers $l,m$ such that
\[
al + p_im = 1
\] Multiplying both sides by $b$, we find that
\[
abl + p_ibm = b
\] Then $p_i$ divides $abl$ and $p_ibm$, so it also divides $abl + p_ibm = b$. This proves that $p_i$ divides either $a$ or $b$. If $p_i$ divides $a$, then $a$ must have all the copies of $p_i$ in its prime factorization (if $b$ contained any copies of $p_i$ in its prime factorization, then $(a,b)$ would be greater than $1$, a contradiction). Since this is true for every prime $p_i$, we may conclude that $a$ and $b$ are equal to a product of terms from the list $p_1^{2\beta_1}, \ldots, p_k^{2\beta_k}$. Since $p_i^{2 \beta_i} = (p_i^{\beta_i})^2$, we may conclude that there exist constants $c_1$ and $c_2$ such that $a = c_1^2$ and $b = c_2^2$.
\newpage
\section*{Problem 2}
First, we must prove that exactly one of $x$ and $y$ is even and the other is odd. For the sake of contradiction, we may first suppose that $x$ and $y$ are both even. We may write $x = 2p$ and $y = 2q$ for integers $p$ and $q$. Then, we have
\[
x^2 + y^2 = z^2 \implies (2p)^2 + (2q)^2 = z^2 \implies 4(p^2+q^2) = z^2 \implies 4 \mid z^2 \implies 2 \mid z
\] Then, we have $(x,y,z) \geq 2$, which is false by assumption. Next, we may suppose that $x$ and $y$ are both odd. Thus, we may write $x = 2p+1$ and $y = 2q + 1$ for some positive integers $p$ and $q$. Notice that
\[
x^2+y^2 = (2p+1)^2 + (2q+1)^2 = 4p^2 + 4p + 1 + 4q^2 + 4q + 1 = 2(2p^2+2p+2q^2+2q+1) = z^2
\] Thus we find that $2 \mid z^2$, which implies that $2 \mid z$. We may write $z = 2t$. Then, we find that
\[
2(2p^2+2p+2q^2+2q+1) = 4t^2
\] from which we obtain
\[
2p^2+2p+2q^2+2q+1 = 2t^2
\] This says that an odd number is equal to an even number, which is false. Therefore, we know that one of $x$ and $y$ must be even and the other must be odd.
\\
Next, we must show that
\[
\bigg(\frac{z+y}{2},\frac{z-y}{2}\bigg) = 1
\] Note that $(z+y)/2$ and $(z-y)/2$ are integers because $z$ and $y$ are both odd. Suppose that $d$ is a common divisor of $\frac{z+y}{2}$ and $\frac{z-y}{2}$. Then it must divide their sum and difference, so $d \mid z$ and $d \mid y$. We claim that $y$ and $z$ are pairwise prime. If this were not the case, then there would be some prime $p$ such that $p\mid y$ and $p \mid z$. From the relation $x^2+y^2 = z^2$, we would have $p \mid x^2$, so that $p \mid x$, which would imply that $x,y,z$ are not relatively prime. This is a contradiction. Thus, $y$ and $z$ are pairwise prime. Since $d$ divides pairwise prime numbers, we have $d=1$. This means that
\[
\bigg(\frac{z+y}{2},\frac{z-y}{2}\bigg) = 1
\]
\\
Now, we know that 
\[
\bigg(\frac{x}{2}\bigg)^2 = \bigg(\frac{z+y}{2}\bigg)\bigg(\frac{z-y}{2}\bigg)
\] and we just established that
\[
\bigg(\frac{z+y}{2},\frac{z-y}{2}\bigg) = 1
\] Appealing to problem $1$, there must be integers $m$ and $n$ such that
\[
\frac{z+y}{2} = m^2
\] and
\[
\frac{z-y}{2} = n^2
\] Adding $m^2$ and $n^2$, we obtain
\[
z = m^2 + n^2
\] Subtracting $n^2$ from $m^2$ yields
\[
y = m^2 - n^2
\] Finally, we have
\[
x^2 = z^2-y^2 = m^4 + 2m^2n^2 + n^4 - (m^4 - 2m^2n^2 + n^4) = 4m^2n^2 
\] so that
\[
x = 2mn
\]
\newpage
\section*{Problem 3}
We will prove this result by induction. First, let us suppose that $N = (4n_1+1)(4n_2+1)$, where $n_1$ and $n_2$ are integers. We obtain
\[
(4n_1+1)(4n_2+1) = 16n_1n_2 + 4n_1 + 4n_2 + 1 = 4(4n_1n_2 + n_1 + n_2) + 1
\] Now, let us suppose that $N$ is the product of the primes $4n_1+1,\ldots,4n_{k+1}+1$. By our induction hypothesis, we know that
\[
(4n_1+1)\cdots(4n_k+1) = 4m+1
\] for some integer $m$. Then, we find that
\[
(4m+1)(4n_{k+1}+1) = 16mn_{k+1} + 4m + 4n_{k+1} + 1 = 4(4mn_{k+1}+m+n_{k+1})+1
\] By induction, we have shown that if $N$ is a product of any number of primes of the form $4n+1$, then $N = 4M+1$, where $M$ is an integer.
\newpage
\section*{Problem 4}
By the Fundamental Theorem of Arithmetic, we may factor the number $N= 4p_1\cdots p_k + 3$ into primes as follows:
\[N = q_1\cdots q_j\]
First, we claim that some $q_i$ is of the form $4n+3$. Suppose this were not true. Then every $q_i$ must be of the form $4n+1$, so their product would also be of the form $4M+1$ by Problem $3$. However, it is evident that $N$ is of the form $4M+3$, so there must exist some prime $q_i$ of the form $4n+3$, which we may denote $q$. Next, we claim that this $q$ is not equal to any of $p_1,\ldots,p_k$. Notice that none of the primes $p_1,\ldots,p_k$ divide $N$. If some $p_i$ did divide $N$, then it would also have to divide $3= N - p_1\cdots p_k$, which cannot happen because we assumed that $p_1,\ldots,p_k$ were all greater than $3$. Since $q$ does divide $N$, it is evident that $q$ is not equal to any of the primes $p_1,\ldots,p_k$.
\newpage
\section*{Problem 5}
Suppose there were only finitely many primes $p_1,\ldots,p_{k}$ of the form $4n+3$. As in problem $4$, we may consider the number $N = 4p_1\cdots p_k + 3$. In problem $4$, we proved that $N$ has some prime factor $q$ that is of the form $4n+3$ that is not equal to any of the primes $p_1,\ldots,p_k$. This contradicts the assumption that $p_1,\ldots,p_k$ constituted the complete list of primes of the form $4n+3$. Thus there must be infinitely many primes of the form $4n+3$.
\newpage
\section*{Problem 6}
Suppose that there were only finitely many primes $p_1,\ldots,p_k$ of the form $6n+5$. Consider the number
\[
N = 6p_1\cdots p_k + 5
\] By the Fundamental Theorem of Arithmetic, we may factorize $N$ into primes as follows:
\[
N = q_1\ldots q_j
\] We claim that one of these primes $q_i$ must be of the form $6n+5$. For the sake of contradiction, suppose that all of the primes $q_1,\ldots,q_j$ were of the form $6n+1$. Then their product would also be of the form $6n+1$. To see this, consider the integers $6n+1$ and $6m+1$. Their product is
\[
(6n+1)(6m+1) = 36mn + 6n + 6m + 1 = 6(6mn + n + m) + 1
\] By induction, it is evident that the product $q_1\cdots q_j$ would be of the form $6n+1$ which contradicts the fact that $N$ is of the form $6n+5$. Thus, there must exist some prime $q$ of the form $6n+5$ in the prime factorization of $N$. Next, we claim that $q$ is not equal to any prime in the list $p_1,\ldots,p_k$. Notice that $q$ divides $N$ by construction, but none of the primes $p_1,\ldots,p_k$ divide $N$. If any of these primes did divide $N = 6p_1\cdots p_k + 5$, then it would also have to divide $N - 6p_1\cdots p_k = 5$, which cannot happen. Since $q$ divides $N$ but none of the primes $p_1,\ldots,p_k$ divide $N$, we know that $q$ is not equal to any of the primes $p_1,\ldots,p_k$. Thus, we may conclude that there must be infinitely many primes of the form $6n+5$.
\newpage
\section*{Problem 7}
For the sake of contradiction, suppose that 
\[
S_n = 1 + \frac{1}{2} + \cdots + \frac{1}{n}
\]
 was an integer. Then the product of $S_n$ with any other integer must also be an integer (since integers are closed under multiplication). We aim to construct an integer whose product with $S_n$ is not an integer. Let $k$ be the largest integer such that $2^k \leq n$. Let $N$ be equal to $2^{k-1} \cdot p$ (where $p$ is the product of all the odd integers less than $n$). We claim that for every integer $m$ that is less than or equal to $n$ and not equal to $2^{k}$, $N/m$ is an integer. Let $m$ be an arbitrary integer less than or equal to $n$ and not equal to $2^k$. We may write $m = 2^a \cdot b$, where $b$ is odd. First, we claim that $b$ divides $p$. This is evident because $b$ is an odd number less than $n$ so it divides $p$ by the definition of $p$ (recall that $p$ was defined to be the product of all odd integers less than $n$). Next, we claim that $a \leq k-1$. Suppose that $a \geq k$. If $b = 1$ and $a = k$, then $m = 2^k$, contradicting our assumptions on $m$. If $b>1$ or $a > k$, then $m = 2^a \cdot b \geq 2^{k+1} > n$, again contradicting our assumptions about $m$. Thus, we may deduce that $a \leq k - 1$. Now, we note that $2^a \mid 2^{k-1}$ and $b \mid p$, so $m = 2^a \cdot b \mid 2^{k-1} \cdot p = N$. Let us consider $N/2^k$. This is equal to $2^{k-1}p/2^k = p/2$. Since $p$ is odd, this is not an integer. Notice that
 \[
 NS_n = \sum_{j=1}^n \frac{N}{j} = \sum_{j\neq 2^k} \frac{N}{j} + \frac{N}{2^k} = \sum_{j\neq 2^k} \frac{N}{j} + \frac{p}{2}
 \] Notice that the first sum is an integer, but $p/2$ is not. Thus, $NS_n$ is not an integer. Since $N$ is an integer, we must conclude that $S_n$ is not an integer.
 \end{document} 
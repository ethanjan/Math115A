\documentclass[12pt]{article}
 
\usepackage[margin=1in]{geometry}
\usepackage{amsmath,amsthm,amssymb}
\usepackage{mathtools}
\DeclarePairedDelimiter{\ceil}{\lceil}{\rceil}
%\usepackage{mathptmx}
\usepackage{accents}
\usepackage{comment}
\usepackage{graphicx}
\usepackage{IEEEtrantools}
 \usepackage{float}
 
\newcommand{\N}{\mathbb{N}}
\newcommand{\Z}{\mathbb{Z}}
\newcommand{\R}{\mathbb{R}}
\newcommand{\Q}{\mathbb{Q}}
\newcommand*\conj[1]{\bar{#1}}
\newcommand*\mean[1]{\bar{#1}}
\newcommand\widebar[1]{\mathop{\overline{#1}}}


\newcommand{\cc}{{\mathbb C}}
\newcommand{\rr}{{\mathbb R}}
\newcommand{\qq}{{\mathbb Q}}
\newcommand{\nn}{\mathbb N}
\newcommand{\zz}{\mathbb Z}
\newcommand{\aaa}{{\mathcal A}}
\newcommand{\bbb}{{\mathcal B}}
\newcommand{\rrr}{{\mathcal R}}
\newcommand{\fff}{{\mathcal F}}
\newcommand{\ppp}{{\mathcal P}}
\newcommand{\eps}{\varepsilon}
\newcommand{\vv}{{\mathbf v}}
\newcommand{\ww}{{\mathbf w}}
\newcommand{\xx}{{\mathbf x}}
\newcommand{\ds}{\displaystyle}
\newcommand{\Om}{\Omega}
\newcommand{\dd}{\mathop{}\,\mathrm{d}}
\newcommand{\ud}{\, \mathrm{d}}
\newcommand{\seq}[1]{\left\{#1\right\}_{n=1}^\infty}
\newcommand{\isp}[1]{\quad\text{#1}\quad}

\DeclareMathOperator{\imag}{Im}
\DeclareMathOperator{\re}{Re}
\DeclareMathOperator{\diam}{diam}
\DeclareMathOperator{\Tr}{Tr}

\def\upint{\mathchoice%
    {\mkern13mu\overline{\vphantom{\intop}\mkern7mu}\mkern-20mu}%
    {\mkern7mu\overline{\vphantom{\intop}\mkern7mu}\mkern-14mu}%
    {\mkern7mu\overline{\vphantom{\intop}\mkern7mu}\mkern-14mu}%
    {\mkern7mu\overline{\vphantom{\intop}\mkern7mu}\mkern-14mu}%
  \int}
\def\lowint{\mkern3mu\underline{\vphantom{\intop}\mkern7mu}\mkern-10mu\int}




\newenvironment{theorem}[2][Theorem]{\begin{trivlist}
\item[\hskip \labelsep {\bfseries #1}\hskip \labelsep {\bfseries #2.}]}{\end{trivlist}}
\newenvironment{lemma}[2][Lemma]{\begin{trivlist}
\item[\hskip \labelsep {\bfseries #1}\hskip \labelsep {\bfseries #2.}]}{\end{trivlist}}
\newenvironment{exercise}[2][Exercise]{\begin{trivlist}
\item[\hskip \labelsep {\bfseries #1}\hskip \labelsep {\bfseries #2.}]}{\end{trivlist}}
\newenvironment{problem}[2][Problem]{\begin{trivlist}
\item[\hskip \labelsep {\bfseries #1}\hskip \labelsep {\bfseries #2.}]}{\end{trivlist}}
\newenvironment{question}[2][Question]{\begin{trivlist}
\item[\hskip \labelsep {\bfseries #1}\hskip \labelsep {\bfseries #2.}]}{\end{trivlist}}
\newenvironment{corollary}[2][Corollary]{\begin{trivlist}
\item[\hskip \labelsep {\bfseries #1}\hskip \labelsep {\bfseries #2.}]}{\end{trivlist}}

\newenvironment{solution}{\begin{proof}[Solution]}{\end{proof}}
 
\begin{document}
 
% --------------------------------------------------------------
%                         Start here
% --------------------------------------------------------------
\title{Math 115A Final Exam}
\author{Ethan Martirosyan}
\date{\today}
\maketitle
\hbadness=99999
\hfuzz=50pt
\section*{Problem 1}
\subsection*{Part (i)}
We note that
\[
\bigg(\frac{105}{191}\bigg) = \bigg(\frac{3}{191}\bigg)\bigg(\frac{5}{191}\bigg)\bigg(\frac{7}{191}\bigg)
\] First, we will compute 
\[
\bigg(\frac{3}{191}\bigg)
\] Notice that $3 \equiv 3 \pmod{4}$ and $191 \equiv 3 \pmod{4}$. Appealing to the quadratic reciprocity law, we have
\[
\bigg(\frac{3}{191}\bigg) = -\bigg(\frac{191}{3}\bigg) = -\bigg(\frac{2}{3}\bigg) = -(-1)^{(3^2-1)/8} = -(-1) = 1
\] Next, we compute
\[
\bigg(\frac{5}{191}\bigg)
\] Notice that $5 \equiv 1\pmod{4}$. By the quadratic reciprocity law, we have
\[
\bigg(\frac{5}{191}\bigg) = \bigg(\frac{191}{5}\bigg) = \bigg(\frac{1}{5}\bigg) = 1
\] Finally, we compute
\[
\bigg(\frac{7}{191}\bigg)
\] Notice that $7 \equiv 3 \pmod{4}$ and $191 \equiv 3 \pmod{4}$. By the quadratic reciprocity law, we have
\[
\bigg(\frac{7}{191}\bigg) = -\bigg(\frac{191}{7}\bigg) = -\bigg(\frac{2}{7}\bigg) = - (-1)^{(7^2-1)/8} = -(-1)^{6} = -1
\] Substitution then yields
\[
\bigg(\frac{105}{191}\bigg) = \bigg(\frac{3}{191}\bigg)\bigg(\frac{5}{191}\bigg)\bigg(\frac{7}{191}\bigg) = 1 \cdot 1 \cdot - 1 = - 1
\]
\newpage
\subsection*{Part (ii)}
Notice that
\[
\bigg(\frac{56}{101}\bigg) = \bigg(\frac{7}{101}\bigg) \bigg(\frac{8}{101}\bigg) = \bigg(\frac{7}{101}\bigg) \bigg(\frac{2^3}{101}\bigg) = \bigg(\frac{7}{101}\bigg) \bigg(\frac{2}{101}\bigg)^3 = \bigg(\frac{7}{101}\bigg) \bigg(\frac{2}{101}\bigg)
\] First, we will compute
\[
\bigg(\frac{7}{101}\bigg)
\] Since $101 \equiv 1 \pmod{4}$, we may appeal to the quadratic reciprocity law to deduce that
\[
\bigg(\frac{7}{101}\bigg) = \bigg(\frac{101}{7}\bigg) = \bigg(\frac{3}{7}\bigg)
\] Since $3 \equiv 3 \pmod{4}$ and $7 \equiv 3 \pmod{4}$, we may appeal to the quadratic reciprocity law to find
\[
\bigg(\frac{3}{7}\bigg) = -\bigg(\frac{7}{3}\bigg) = -\bigg(\frac{1}{3}\bigg) = -1
\] Next, we will compute 
\[
\bigg(\frac{2}{101}\bigg)
\] Notice that $101 \equiv 5 \pmod{8}$ so that
\[
\bigg(\frac{2}{101}\bigg) = - 1
\] Substitution then yields
\[
\bigg(\frac{56}{101}\bigg) = \bigg(\frac{7}{101}\bigg) \bigg(\frac{2}{101}\bigg) = (-1)\cdot (-1) = 1
\]
\newpage
\subsection*{Part (iii)}
Notice that
\[
\bigg(\frac{106}{89}\bigg) = \bigg(\frac{17}{89}\bigg) 
\] since $106 \equiv 17 \pmod{89}$. Because $17 \equiv 1 \pmod{4}$, the quadratic reciprocity law informs us that
\[
\bigg(\frac{17}{89}\bigg) = \bigg(\frac{89}{17}\bigg)
\] Because $89 \equiv 4 \pmod{17}$, we have
\[
\bigg(\frac{89}{17}\bigg) = \bigg(\frac{4}{17}\bigg) = \bigg(\frac{2^2}{17}\bigg) = \bigg(\frac{2}{17}\bigg)^2 = 1
\]
\newpage
\section*{Problem 2}
\subsection*{Part (i)}
First, we compute $(14, 31)$ as follows:
\begin{align*}
31 & = 14 \cdot 2 + 3\\
14 &= 3 \cdot 4 + 2\\
3 &= 2 \cdot 1 + 1\\
2 &= 2 \cdot 1
\end{align*} This shows that $(14,31) = 1$. Thus, we know that this congruence has exactly one solution. Now, we have
\begin{align*}
1= 3 - 2 = 3 - (14 - 3 \cdot 4) = 5 \cdot 3 - 14 = 5 \cdot (31 - 14 \cdot 2) - 14 = 5 \cdot 31 - 11 \cdot 14
\end{align*} From this, we find that
\[
-11\cdot 14 \equiv 1 \pmod{31}
\] so that
\[
-33 \cdot 14 \equiv 3 \pmod{31}
\] Notice that $-33 \equiv 29 \pmod{31}$, so the solution of $14x \equiv 3 \pmod{31}$ is $29 \pmod{31}$.
\newpage
\subsection*{Part (ii)}
First, we will compute $(35,15)$ as follows:
\begin{align*}
35 &= 15 \cdot 2 + 5\\
15 &= 5 \cdot 3
\end{align*} so that $(35,15) = 5$. However, we note that $5 \nmid 9$, so the congruence $15x \equiv 9 \pmod{35}$ has no solution.
\newpage
\subsection*{Part (iii)}
First, we may compute $(35,56)$ as follows:
\begin{align*}
56 &= 35 \cdot 1 + 21\\
35 &= 21 \cdot 1 + 14\\
21 &= 14 \cdot 1 + 7\\
14 &= 7 \cdot 2
\end{align*} so that $(35,56) = 7$. Since $7 \mid 14$, we know that there are $7$ solutions. Dividing the congruence $35x \equiv 14 \pmod{56}$ by $7$ yields $5x \equiv 2 \pmod{8}$. Performing the Euclidean Algorithm, we obtain
\begin{align*}
8 &= 5 \cdot 1 + 3\\
5 &= 3 \cdot 1 + 2\\
3 &= 2\cdot 1 + 1\\
2 & = 2 \cdot 1
\end{align*} Then, we have
\begin{align*}
1 = 3 - 2 = 3 - (5 - 3) = 2\cdot 3 - 5 = 2\cdot (8-5) - 5 = - 3\cdot 5 + 2\cdot 8
\end{align*} so that $5 \cdot -3  \equiv 1 \pmod{8}$. This informs us that $5 \cdot -6 \equiv 2 \pmod{8}$. Since $-6 \equiv 2 \pmod{8}$, we find that $5 \cdot 2 \equiv 2 \pmod{8}$ so that $x = 2$ is a solution of the congruence $5x \equiv 2 \pmod{8}$. Thus it is also a solution of the congruence $35x \equiv 14 \pmod{56}$. The other solutions can be found by adding multiples of $8$. That is, the solutions of $35x \equiv 14 \pmod{56}$ are $x \equiv 2, 10, 18, 26, 34, 42, 50 \pmod{56}$.
\newpage
\section*{Problem 3}
\subsection*{Part (i)}
First, we note that 
\[
\bigg[ \frac{x}{n} \bigg] = \sum_{m \leq x/n} 1
\] because \([x/n]\) counts the number of positive integers less than or equal to \(x/n\). Using this, we find that
\[
\sum_{n \leq x} \bigg[ \frac{x}{n} \bigg] = \sum_{n \leq x} \sum_{m \leq x/n} 1 = \sum_{n \leq x} \sum_{nm \leq x} 1
\] Notice that we are summing over all pairs \((n,m)\) of positive integers such that \(nm \leq x\). Letting \(l = mn\) and interchanging the order of summation, we obtain
\[
\sum_{n \leq x} \sum_{nm \leq x} 1 = \sum_{l \leq x} \sum_{n \mid l} 1 = \sum_{l \leq x} \tau(l)
\]
\newpage
\subsection*{Part (ii)}
We may use the following facts: 
\[
\bigg [ \frac{x}{n} \bigg ] = \frac{x}{n} + O(1)
\] and
\[
\sum_{n \leq x} \frac{1}{n} = \log x + O(1)
\] Now, we note that
\begin{align*}
&\sum_{n \leq x} \tau(n) = \sum_{n \leq x} \bigg[ \frac{x}{n} \bigg] = \sum_{n \leq x} \bigg( \frac{x}{n} + O(1) \bigg) = \sum_{n \leq x} \frac{x}{n} + O(x) = x \sum_{n \leq x} \frac{1}{n} + O(x) \\
&= x(\log x + O(1)) + O(x) = x\log x + O(x) + O(x) = x\log x + O(2x)\\
& = x\log x + O(x)
\end{align*} by the properties of \(O\)-notation.
\newpage
\section*{Problem 4}
First, we claim that the set
\[
A = \{an+b \mid 1 \leq n \leq m\}
\] is a complete system of residues modulo \(m\). To prove this, we will show that \(\vert A \vert = m\) and that \(A\) is incongruent modulo \(m\). It is evident that \(\vert A \vert = m\) because there are \(m\) numbers between \(1\) and \(m\). Next, we claim that \(A\) is incongruent modulo \(m\). Suppose that 
\[
an_1 + b \equiv an_2 + b \pmod{m}
\]
 for some $n_1$ and $n_2$ satisfying \(1 \leq n_1, n_2 \leq m\). Subtracting \(b\) yields
 \[
an_1 \equiv an_2 \pmod{m} 
 \] Since \((a,m) = 1\), we may divide by \(a\) to obtain
 \[
 n_1 \equiv n_2 \pmod{m}
 \] This implies that \(n_1 - n_2\) is divisible by \(m\). Since \(\vert n_1 - n_2 \vert < m\), we find that \(n_1 = n_2\) so that \(an_1 + b = an_2 + b\), thus proving that the set \(A\) is incongruent modulo $m$. This shows that \(A\) is a complete system of residues modulo \(m\). Notice that the set \(B = \{0,1,\ldots,m-1\}\) is also a complete system of residues modulo \(m\). Finally, we note that the function \(\{x\}\) is of period \(1\). By a theorem we proved in class, we know that
\[
\sum_{x \in A} \bigg\{ \frac{x}{m}\bigg\} = \sum_{y \in B} \bigg\{ \frac{y}{m}\bigg\}
\]
so that 
 \[
 \sum_{n=1}^{m} \bigg\{ \frac{an+b}{m} \bigg\} = \sum_{n=0}^{m-1} \bigg\{\frac{n}{m}\bigg\} = \sum_{n=0}^{m-1} \frac{n}{m} = \frac{1}{m} \sum_{n=0}^{m-1} n = \frac{1}{m} \cdot \frac{(m-1)m}{2} = \frac{m-1}{2}
 \]
\newpage
\section*{Problem 5}
\subsection*{Part (i)}
The set \(S\) is a reduced system of residues modulo \(p\). In class, we proved that \(S\) contains \((p-1)/2\) quadratic residues modulo \(p\) and \((p-1)/2\) quadratic non-residues modulo \(p\). By definition, we know that
\[
\bigg( \frac{s}{p} \bigg) = 1
\] if $s$ is a quadratic residue modulo $p$ and
\[
\bigg( \frac{s}{p} \bigg) = -1
\] if $s$ is a non-quadratic residue modulo $p$.
Thus, we find that
\[
\sum_{s \in S} \bigg( \frac{s}{p} \bigg) = \frac{p-1}{2} - \frac{p-1}{2} = 0
\]
\newpage
\subsection*{Part (ii)}
Let $R = \{1,2,\ldots,p-1\}$ be a reduced system of residues modulo $p$. First, we claim that
\[
\sum_{s\in S} \bigg(\frac{1+s}{p}\bigg) = \sum_{r\in R} \bigg(\frac{1+r}{p}\bigg)
\] To show this, we note that $S$ and $R$ are both reduced systems of residues modulo $p$. Thus, we know that for every $s \in S$, there is exactly one $r \in R$ such that $s \equiv r \pmod{p}$. Then, we have $s + 1 \equiv r + 1\pmod{p}$ so that
\[
\bigg(\frac{s+1}{p}\bigg) = \bigg(\frac{r+1}{p}\bigg)
\] Summing over $s \in S$ and $r \in R$ then yields
\[
\sum_{s\in S} \bigg(\frac{1+s}{p}\bigg) = \sum_{r\in R} \bigg(\frac{1+r}{p}\bigg)
\] Notice that
\[
\sum_{r\in R} \bigg(\frac{1+r}{p}\bigg) = \bigg(\frac{2}{p}\bigg) + \bigg(\frac{3}{p}\bigg) + \cdots + \bigg(\frac{p-1}{p}\bigg) + \bigg(\frac{p}{p}\bigg)
\] By assumption, we know that
\[
\bigg(\frac{p}{p}\bigg) = 0
\] so that  
\[
\bigg(\frac{2}{p}\bigg) + \bigg(\frac{3}{p}\bigg) + \cdots + \bigg(\frac{p-1}{p}\bigg) + \bigg(\frac{p}{p}\bigg) = \bigg(\frac{2}{p}\bigg) + \bigg(\frac{3}{p}\bigg) + \cdots + \bigg(\frac{p-1}{p}\bigg)
\]
Since
\[
\bigg(\frac{1}{p}\bigg) = 1
\] we may write
\[
\bigg(\frac{2}{p}\bigg) + \bigg(\frac{3}{p}\bigg) + \cdots + \bigg(\frac{p-1}{p}\bigg) = \bigg(\frac{1}{p}\bigg) + \bigg(\frac{2}{p}\bigg) + \bigg(\frac{3}{p}\bigg) + \cdots + \bigg(\frac{p-1}{p}\bigg) - 1
\] Since $R = \{1,\ldots,p-1\}$, we have
\[
\bigg(\frac{1}{p}\bigg) + \bigg(\frac{2}{p}\bigg) + \bigg(\frac{3}{p}\bigg) + \cdots + \bigg(\frac{p-1}{p}\bigg) - 1 = \sum_{r\in R} \bigg( \frac{r}{p}\bigg) - 1 = 0 - 1 = -1
\] by part $(i)$.
\newpage
\section*{Problem 6}
Let $R = \{1,\ldots, p - 1\}$ and let $r \in R$. We may consider the congruence
\[
xr \equiv 1 \pmod{p}
\] We claim that this congruence has a unique solution $x$ modulo $p$. Since $(r,p) = 1$, there must exist integers $a$ and $b$ such that
\[
ar + bp = 1
\] so that $ar \equiv 1 \pmod{p}$. Thus $x = a$ is a solution to this congruence. Next, we claim that this solution is unique. Suppose that $x_1r \equiv x_2r \equiv 1 \pmod{p}$. Then, we have
\[
(x_1-x_2)r \equiv 0 \pmod{p}
\] Thus $(x_1 - x_2)r$ is divisible by $p$. Since $r$ is not divisible by $p$, we find that $x_1 - x_2$ is divisible by $p$ so that $x_1 \equiv x_2 \pmod{p}$. This shows that the solution is unique. Now, we may define $f: R \rightarrow R$ as follows: $f(r) \cdot r \equiv 1 \pmod{p}$. We claim that $f$ is bijective. First, we will show that $f$ is injective. Suppose that $f(x_1) = f(x_2)$ for $x_1,x_2 \in R$. Then, we have
\[
f(x_1)x_1 \equiv 1 \equiv f(x_2)x_2 \pmod{p}
\] By substituting $f(x_1) = f(x_2)$ into the previous congruence, we have
\[
f(x_1)x_1 \equiv f(x_1)x_2 \pmod{p}
\] Since $(f(x_1), p) = 1$, we deduce that
\[
x_1 \equiv x_2 \pmod{p}
\] Thus $x_1 - x_2$ is divisible by $p$. Since $\vert x_1 - x_2 \vert < p$, we must have $x_1 = x_2$, so $f$ is injective. Next, we claim that $f$ is surjective. Let $a \in R$. Above, we proved that there exists some $x \in R$ such that $ax \equiv 1 \pmod{p}$. By definition, $x = f(a)$. Thus $f$ is surjective. This means that $f$ is bijective.
Now, we note that
\[
\bigg(\frac{f(r)}{p} \bigg)^2 = 1
\] so that
\[
\bigg(\frac{r(r+k)}{p}\bigg) = \bigg(\frac{f(r)}{p} \bigg)^2 \bigg(\frac{r(r+k)}{p}\bigg) =  \bigg(\frac{f(r)}{p} \bigg) \bigg(\frac{f(r)}{p} \bigg) \bigg(\frac{r}{p}\bigg) \bigg(\frac{r+k}{p}\bigg) = \bigg(\frac{f(r)r}{p}\bigg)\bigg(\frac{f(r)r + kf(r)}{p}\bigg)
\] Note that
\[
 \bigg(\frac{f(r)r}{p}\bigg) = \bigg(\frac{1}{p}\bigg) = 1
\] since $f(r)r \equiv 1 \pmod{p}$. Similarly, we have
\[
\bigg(\frac{f(r)r + kf(r)}{p}\bigg) = \bigg(\frac{1 + kf(r)}{p}\bigg)
\] since $f(r)r + kf(r) \equiv 1 + kf(r) \pmod{p}$. Using these two facts, we find that 
\[
\bigg(\frac{f(r)r}{p}\bigg)\bigg(\frac{f(r)r + kf(r)}{p}\bigg) = \bigg(\frac{1 + kf(r)}{p}\bigg) 
\] Since $f$ is bijective, $\{f(r) \mid r \in R\} = R$ is a reduced system of residues modulo $p$. Because $(k,p) = 1$, we find that $\{kf(r) \mid r\in R\}$ is also a reduced system of residues modulo $p$. Thus, we deduce that
\[
\sum_{r=1}^{p-1} \bigg(\frac{r(r+k)}{p}\bigg) = \sum_{r \in R} \bigg( \frac{r(r+k)}{p} \bigg) = \sum_{r \in R} \bigg( \frac{1+kf(r)}{p} \bigg) = \sum_{r \in R} \bigg( \frac{1+r}{p} \bigg) = -1
\] by part $(ii)$ of problem $5$.
\end{document} 